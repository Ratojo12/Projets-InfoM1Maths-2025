\documentclass[a4paper,12pt]{article} % Type de document et taille de police
\usepackage[utf8]{inputenc} % Encodage
\usepackage[T1]{fontenc} % Gestion des polices
\usepackage{lmodern} % Police moderne
\usepackage{geometry} % Marges personnalisées
\usepackage{graphicx} % Pour inclure des images (logo)
\usepackage{fancyhdr} % Pour personnaliser l'en-tête et le pied de page
\usepackage{lipsum} % Pour du texte de remplissage (optionnel)

% Configuration des marges
\geometry{margin=2.5cm}

% Configuration de l'en-tête et du pied de page
\pagestyle{fancy}
\fancyhf{} % Réinitialise l'en-tête et le pied de page
\fancyhead[L]{ % En-tête à gauche
    \includegraphics[width=3cm]{N.png} % Inclure ton logo (ajuste la taille)
}
\fancyhead[R]{ % En-tête à droite
    \textbf{Mon Document Personnalisé} \\ % Titre personnalisé
    \small Date : \today % Date automatique
}
\fancyfoot[C]{\thepage} % Numéro de page au centre du pied de page
\renewcommand{\headrulewidth}{0.4 pt} % Ligne sous l'en-tête

% Début du document
\begin{document}

% Page de titre (optionnelle)
\begin{titlepage}
    \centering
    \includegraphics[width=5cm]{M.png} \\ % Logo sur la page de titre
    \vspace{1 cm}
    \Huge \textbf{Titre du Document} \\
    \vspace{0.5cm}
    \Large Sous-titre ou auteur \\
    \vspace{1cm}
   % \normalsize \today
\end{titlepage}\\
\\\\
\\\\
\\
\\\\
\\\\\\\\\\
\\\\\\\


\\\\\
\\
\\\\\\\\\\\
\section{Introduction}
L’objet des trois premiers chapitres est de donner la description compl`ete
et les propri´et´es du triplet fondamental (Ω, A,P), maintenant adopt´e par tous
les probabilistes pour une premi`ere ad´equation math´ematique de la notion
de hasard. D’abord, il faut d´ecrire les ´ev`enements1 attach´es au ph´enom`ene
consid´er´e o`u intervient le hasard. C’est l’objet de ce pr´esent chapitre. Ces
´ev`enements apparaˆıtront comme des sous-ensembles du premier ´el´ement Ω
du triplet. Pour permettre une manipulation ais´ee des ´ev`enements, nous
introduirons, au chapitre suivant, essentiellement une famille remarquable A
d’´ev`enements, qui ob´eit `a des r`egles alg´ebriques pr´ecises. Dans le troisi`eme
chapitre enfin
\section{Espace probabilise }
On part maintenant d’un espace probabilisable (Ω, A) et on cherche
`a munir les ´ev`enements, c’est-`a-dire les ´el´ements de la tribu A, d’une
pond´eration utilisant pleinement les propri´et´es des tribus. On obtient ainsi un
triplet (Ω, A,P), appel´e espace probabilis´e. L’id´ee de mod´eliser une exp´erience
al´eatoire au moyen d’un tel triplet (non n´ecessairement unique!) a marqu´e
un tournant d´ecisif dans le d´eveloppement du calcul des probabilit´es.


\end{document}
